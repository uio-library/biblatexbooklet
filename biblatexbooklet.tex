\documentclass[11pt,english,a4paper]{article}
\usepackage[utf8]{inputenc}
\usepackage{babel,csquotes}
\usepackage[T1]{fontenc}
\usepackage[pdftex]{graphicx}
\usepackage{relsize}
\usepackage[hyphens]{url}

% illustrasjonene er laget i powerpoint i filen illustrasjoner.ppt
% og kan revideres der.


% her kommer biblatex-tingene
% sjekk evt for definisjoner
% /local/store/localhost/.texmf-ctan-biblatex/ver-2.4/opt/texlive/texmf-local/tex/latex/biblatex  
\usepackage[%
backend=biber,%
style=numeric,%
sorting=none,%
hyperref=true,%
backref,%
date=long,%
urldate=short%
]{biblatex}

\addbibresource{referanser.bib}
% local redefinitions
\DefineBibliographyStrings{english}{%
urlseen={Seen: },
%andre muligheter:
%bibliography = {Bibliografi},
%references = {Referanser},
%editor = {redaktør},
%translator={oversetter},
%page={side},
%pages={sidene},
%and={og},
}

% I will define the prefix for URLs myself
\DeclareFieldFormat{url}{\url{#1}} 
\DeclareUrlCommand\url{\def\UrlLeft{\newline}\def\UrlRight{\newline}%
\urlstyle{sf}} % setter inn passende linjeskift

% biblatex recommends that the hyperref package is loaded after biblatex
\usepackage[colorlinks,allcolors=blue]{hyperref}

% her er mine spesialkommandoer
\newcommand{\kdo}[1]{\texttt{#1}}
%\newcommand{\pil}{$>\ $}
\newcommand{\pil}{ > }
\newcommand{\prikk}{$\bullet\ $}
\newcommand{\en}{EndNote{}}
\newcommand{\bt}{BibTeX{}}
\newcommand{\blt}{B{\smaller[2]IB}\discretionary{-}{}{\kern
    -0.12em}\LaTeX{}}
\newcommand{\underskrift}[1]{\vspace{.3cm}\noindent\textbf{#1}\newline}
\newcommand{\flipp}{\vspace{0.1cm}\newline\indent}
\newcommand{\mnd}{Mandatory}
\title{\blt -- course notes}
\author{Science library. Informatics\\ University of Oslo}
%\date{January 24th, 2014}
\begin{document}

\maketitle{}
\begin{center}
 {\footnotesize Coloured texts are links.}
\end{center}

\tableofcontents
\newpage
\section*{Preface}
This paper deals with the practical exercises in the \blt{}-course and
saves you some note taking during the course. The exercises will help
you build the bibliography given in the article entitled \textit{User
  interface} from Encyclopedia of computer science\cite{jacob00-2}.

% \section{Bibliographies and reference lists}
% You may want to skip this section and go directly to section~\ref{blt}

% \subsection*{What is a bibliography?}
% A bibliography is a list of literature sharing some common
% aspect. This might be a person, a subject, an area (national
% bibliographies) and others.

% \begin{itemize}
% \item Within computer science there are a lot of bibliographies
%   covering the different parts of the field. Many of these
%   bibliographies are available in the service  \textit{The
%     Collection of Computer Science Bibliographies}\cite{achilles}.

% This service contains bibliographies in many categories:
% \textit{Artificial Intelligence, Compiler Technology, Programming
%   Languages and Type Theory, Database Research, Distributed Systems,
%   Networking and Telecommunications, Computer Graphics and Vision,
%   Logic Programming, (Computational) Mathematics, Neural Networks,
%   \ldots}

% All together more than 3~000~000 records (duplicates included).
% \item\textit{Digital Bibliography \& Library Project} (DBLP -see our
%   home page) contains mainly journal articles, conference
%   papers and links to the home pages of computer scientists.

% \item\textit{National bibliographies} are another type of
%   bibliography. They cover all that is published within a country.
% \item\textit{Person bibliographies} usually cover all that is written
%   by and about a certain person.
% \end{itemize}

\subsection*{What is a reference list? Why refer?}
Without a reference list, a thesis is not finished. When working with
research you always base your work more or less on previous works done
by others. If you don't cite them when you do, it will be considered
theft or plagiarism.

The reference list in your thesis should contain information on
\textit{every} document that you either cite, are referring to or from
which you use ideas, definitions, illustrations, graphs, statistics or
other results.

When giving definitions of concepts you should always cite the source,
because this tells the reader what context you operate within.

The body text must contain \textit{citations} which are pointers to
the reference list. The reference list contains all the information
that is necessary to locate and access the source.

Now, how far should you go in citing? This varies from discipline to
discipline. If in doubt discuss this with your supervisor.

What might be considered as common knowledge, like the
fact that the earth moves around the sun\cite{copernicus}, need not be
cited. What is common knowledge in one discipline, might not be in
another. It also depends on for whom you are writing.

You should aim to cite the primary source. A secondary source might
not have understood or cited the primary source correctly. Ensure
yourself that you understand your source and that you express the
source correctly.

\subsection*{The importance of consistent and correct references}
The purpose of the reference list is to show the reader on which
previous research your own work is based and to give the readers of
your work a chance to evaluate it.

This means that references must contain the necessary and sufficient
information elements to identify and locate the described document.

This course deals with the question of making a proper reference list
using the tool \blt.

%\CYRM\CYRO\CYRS\CYRK\CYRV\CYRA

\section{The bib-file and Emacs}\label{blt}
\subsection{Literature}
The IFI departement has published a short guide to \blt\
\cite{langmyhr13}. If you want to dive into it, you may read the
\blt{} package documentation\cite{biblatex}.

% You will find English texts on \bt\ in
% \cite{kopka95,kopka04,syropoulos02} which may be borrowed from the
% library.

A wiki-book on \LaTeX\ includes a useful page on \bt\
\cite{wikibook} with a short  description of \blt.

% \subsection{GUI for BibTeX}
% You will find several tools for handling \bt-references on the
% net. Among these are:
% \begin{itemize}
% \item\textit{JabRef} which you can download from
% \url{http://jabref.sourceforge.net/}. It is written in Java and is
% said to run on Windows, Mac and Linux.
% \item\textit{Zotero} is an elegant add-on for the Firefox web
%   browser. It may be downloaded from \url{http://www.zotero.org}. The
%   add-on may be used for collecting and editing references downloaded
%   from web pages. Zotero may export references in \bt-syntax to a
%   bib-file which may be used with \LaTeX.

% Zotero was also mentioned in the master introduction lecture, you will
% find the slides via\newline
% \url{http://tinyurl.com/ifibib}

% \end{itemize}

\subsection{The bib-file}
References are collected in a file with the extension
\textit{bib}. The connection between the \LaTeX{} file and the
references in the bib-file is established by using identifiers (see
below). 

The bib-file may reside in the same file folder as the \LaTeX{}
file or you may put it in the file folder \verb=~/texmf/bibtex/bib=
where \blt\ always will find it.

\subsection{Emacs}
Create a folder called \textit{biblatexcourse} in your home directory
and locate yourself to this folder. Under Linux it looks like this:

\verb=       > mkdir biblatexcourse =

\verb=       > cd biblatexcourse =

\noindent{}Open a file named \textit{myreferences.bib} in Emacs by giving the
command\footnote{Under Windows, select \textit{emacs.exe} from the program menu}:

\verb=       > emacs myreferences.bib & =

\noindent{}Emacs uses a certain \bt-mode when you open a bib-file. You will see a menu called \textit{Entry-Types}. When you want to
enter a new reference in your bib-file, you choose the reference type from
this menu. The last option in this menu is \bt-dialect. The default selection of document types and fields are \bt, but may be changed to \blt\ by using the last choice in the \textit{Entry-types} menu (from version 24.2 of Emacs). Select \blt.

After chosing a reference type you will see a list of empty fields,
like this (\textit{article in journal} or Ctrl-c Ctrl-e
Ctrl-a): 

{\footnotesize\begin{verbatim}
    @Article{,
      author =       {},
      title =        {},
      journaltitle = {},
      ALTyear =      {},
      ALTdate =      {},
      OPTvolume =    {},
      OPTnumber =    {},
      OPTpages =     {},
      OPTmonth =     {},
      ...
    }
\end{verbatim}}



\subsection*{Identifiers}
Every reference in the bib-file must include a unique identifier. It
must be entered directly following the first curly brace. A comma concludes
the identifier.

As you are going to use this identifier when citing in your main
\LaTeX-document you should construct it easy to memorize, like:
\textit{shneiderman1983} and \textit{olsen1992}.

\subsection*{Fields}
The fields prefixed with \textit{OPT} are optional, the others are
mandatory. Sometimes two or more fields are prefixed with
\textit{ALT}. You must enter data into exactly one of them.

The text in the fields must be surrounded by quotes (\verb=" "=) or curly
braces (\verb={}=). There are two exceptions: you may enter clean
numbers (a four digit
year) and macros. See section~\ref{macros}.

The fields are separated by comma.

\subsection*{The author field}
The name should be entered straight forward or with the last name
first followed by a comma and then the rest of the name.

\blt{} will try to split a straight forward name into four parts:
first name, middle names like \textit{von, van, de, \ldots}, last name
and finally additions like \textit{junior, senior, \ldots}. This
procedure doesn't always succeed, so the safest way is to enter the
last name followed by comma.

Several authors must be separated by the word \textbf{and}.

Initials in names should be enteres with a period and a space, like
this: \texttt{Knuth, D. E.}.

These rules holds true also for other personal name fields: \textbf{editor,
 translator,\ldots}.

In general, you should try to enter the full names and no
initials. The bibliographic style will decide whether the names will
be presented with initials or not. If you do not know the full name, try normalising.

\subsection*{Uppercase letters}
Information in some of the fields will be edited by \bt\ according to the
style you have chosen. Uppercase letters might be converted to
lowercase. This does not happen i \blt. If you want to keep your own layout (e.g. acronyms like ACM
and IEEE or person names in titles) put curly braces around
your text.

\subsection{Macros\label{macros}}
Some information occurs often. It may be journal titles, publisher
names, personal names and so on. To save typing and to ensure
consistency you may use \textit{macros} or \textit{aliases}. The definition of the macro
must appear in the beginning of the bib-file. Examples:

{\footnotesize
\begin{verbatim}
  @string{ben   = "Shneiderman, Ben"}
  @string{ojd   = "Dahl, Ole-Johan"}
  @string{tochi = "ACM Transactions on Computer-Human Interaction"}
  @string{aw    = "Addison-Wesley"}
\end{verbatim}
}

\noindent{}It is not possible to use two or more name macros in the 
author field. You will have to define a macro combining the 
names\footnote{See an example in the tex-file later in this course.}.

When the above macros are defined,  you might enter (without curly braces or quotes):

{\footnotesize\begin{verbatim}
    author = ojd,
    editor = ben,
    journaltitle = tochi,
    publisher = aw,
\end{verbatim}}

\noindent{}The macros will be expanded during the  processing.


\subsection{The most important reference types}

We will only comment on the reference types most used. For each type
we list mandatory fields and some
recommended fields. In most cases the program \textbf{biber} will convert \bt\ data to
\blt\footnote{The databases from which we will fetch references still
  only support the \bt\ format.}.

\underskrift{Article in journal} 
\indent\begin{tabular}{ll}
\mnd:&\textit{author, title, journaltitle, year}\\
%\bt: &\textit{author, title, journal, year}\\
Recommended:& \textit{volume, number, pages}.\\
\end{tabular}

\underskrift{Book}
\indent\begin{tabular}{ll}
\mnd:&\textit{author, title, year/date}\\
When needed: &\textit{edition}\\
%\bt: &\textit{title, publisher, year} and one of\\ &\textit{author} or \textit{editor}\\ 
\end{tabular}

\underskrift{InBook}
\noindent{}A part of a book which forms a self-contained unit with its own title.
\flipp\begin{tabular}{ll}
\mnd:&\textit{author, title, booktitle, year/date}.\\
%\bt: &\textit{author/editor, title, chapter, publisher, year}.\\
Recommended: &\textit{publisher,  pages}.\\ 
\end{tabular}

\underskrift{Collection}
\noindent{}This type is not used in \bt.
\flipp\begin{tabular}{ll}
\mnd:&\textit{editor, title, year/date}\\
Recommended&\textit{publisher}\\ 
\end{tabular}

\underskrift{InCollection -- Chapter in book}
\noindent{}The type is meant for self-contained contributions 
in a collection. The contribution has its own author and title. 
The author refers to the title, the editor to the booktitle.
\flipp\begin{tabular}{ll}
\mnd:&\textit{author, editor, title, booktitle, year/date}. \\
%\bt: &\textit{author, title, booktitle}. \\
Recommended: &\textit{publisher, year}.\\ 
\end{tabular}

\underskrift{Proceedings}
\indent\begin{tabular}{ll}
\mnd:&\textit{title, year/date}. \\
%\bt: &\textit{title, year}. \\
Recommended: &\textit{editor, publisher}.\\ 
\end{tabular}


\underskrift{InProceedings -- conference paper}
\noindent{}This is similar to \verb=@incollection=.
\flipp\begin{tabular}{ll}
\mnd:&\textit{author, title, booktitle, year/date}. \\
%\bt: &\textit{author, title}. \\
Recommended: &\textit{editor, publisher, pages}.\\ 
\end{tabular}

\underskrift{Thesis/Mastersthesis/PhdThesis}
\noindent{}For \textit{thesis} \blt\ uses  the field \texttt{type} to separate
different thesis levels.
\flipp\begin{tabular}{ll}
\mnd:&\textit{author, title, type, institution,
  year/date}. \\ 
%\bt: &\textit{author, title, school, year}. \\ 
\end{tabular}

\underskrift{Report/TechReport}
\noindent{}\blt\ uses \texttt{report} and adds a \textbf{type} field to separate
between types of reports (default value is ''technical report'').
\flipp\begin{tabular}{ll}
\blt:&\textit{author, title, type, institution, year/date}\\
%\bt: &\textit{author, title, institution, year}\\
Recommended: & \textit{number}(report number)\\ 
\end{tabular}


\subsection{The crossref option}
If you have several references from the same collection or conference
it would be rather cumbersome to enter the shared information
concerning the collection or the conference.

The solution is to enter a reference (book, collection or proceedings)
that covers the whole document with the shared information, like the book
title.

The references concerning the parts (chapter, paper) might then
include the field \textbf{crossref} containing the identifier of the
shared document. 

\textbf{Biber} will then merge the information from the part and the
whole (e.g. \textit{inBook/Book}, \textit{inProceedings/Proceedings}
and \textit{collection/inCollection}). See figure~\ref{crossref}.
\begin{figure}
\begin{center}
\includegraphics[width=10cm]{./chapterinbook.jpg}
\caption{Using the reference types inBook and Book and the
  \textbf{crossref} field.}\label{crossref}
\end{center}
\end{figure}

\subsection{URLs}\label{biblatex-url}
All reference types in \blt\ may include the field \textit{url}. When
using this one should also use the field \textit{urldate} to show when
this url was last accessed. Date is important by several
reasons. First of all: web pages have a tendency to change over time,
you should document which version (date) you are referring to. Second:
the documents often change their URLs.
The url-date will be prefixed by the text \textbf{visited on} by
default (contained in a variable called \texttt{urlseen}. You may
change this text by the procedure described in section~\ref{tekster}
 on page~\pageref{tekster}.

It is also recommended that you keep a printed copy of the page for
later control. 

% In the reference list the URL will by default be prefixed by the text
% \textbf{URL:}. You may change this by including the following command
% in the preamble:

% \begin{verbatim}
%      \DeclareFieldFormat{url}{yourOwnText\url{#1}} 
% \end{verbatim}

\subsection{The date format}
The date format depends on the document language and the format. You
may define the format as options to the \blt-package. In this case you
may differentiate between \textbf{date}, \textbf{urldate} and other
dates. More on this issue in \cite[][50--53]{biblatex}.

\section{Exercise 1: Building the bib-file} We will now import some
references from external sources and enter one by hand. The references
are found on the accompanying sheet which is also the reference
section of this document.  The encyclopedia bibliography contains 11
references
\cite{hutchins1986,olsen1992,johnson1989,shneiderman1983,hartson1989,jacob1986,myers1995,foley1990,foley1987,shneiderman1992,stephenson1999}. We
shall take care of a few examples and then download a readymade
bib-file.

Usually you will download the reference in parallel with downloading
the full document. \textit{Make it a habit to fetch the reference data
alongside fetching the document.} It will save you a lot of work later on.


\begin{itemize}
\item\textbf{Hutchins et al., 1986
    \cite{hutchins1986}}\newline\textit{Google Scholar}\newline Before
  searching, do this: select \textit{Settings}
  from the horizontal menu. Scroll down to the section on
  \textit{Bibliography manager}. Tick off \textit{Show links to import
    citations into} and select \bt\ in the pull down menu. Save the
  preferences. 

Search for \textbf{Hutchins Hollan Norman} in Google
  Scholar via \href{https://www.ub.uio.no/informatics}{Library home page}.\footnote{\url{https://www.ub.uio.no/informatics}}

  Take a look at the first hit in the list. Click the link
  \textit{Import into \bt}. You will then see the reference with
  \bt-syntax. Is this the reference we were looking
  for?\footnote{No. Why not?} Study other reference candidates down
  the page. Do you notice any peculiarities?\footnote{A lot of
    citations of the relevant document are listed.} The 26 versions do not contain a reference from 1986.

  We will now try to find the reference in the \textit{Collection of
    Computer Science Bibliographies} (CCSB).\cite{achilles}

  Click \textbf{Search} in the horizontal menu in the start
  page. Enter \textit{Hutchins} in the search field and select
  \textit{author} as qualifier and \textit{1986} as publication year.

  The third hit is the one we're looking for. To the right you will
  see two links: \textit{\bt} and \textit{5 duplicates}. Study the
  four duplicates and go back to take a look at the \bt-link. Copy the
  reference into your bib-file.

\item\textbf{Olsen, 1992 \cite{olsen1992}}\newline Go back to Google
  Scholar and try to find this reference by searching for
  \textit{olsen user interface management}. Click \textit{Import into
    \bt} link. Is it OK?\footnote{Yes.}. Copy the reference to the
  bib-file.

\item\textbf{Your own master thesis}\newline Select the correct reference
  type from the menu and enter data in the mandatory fields. Enter
  the command \kdo{Ctrl-c Ctrl-c} to clean up the record and generate
  an identifier.

\item\textbf{Johnson et al., 1989 \cite{johnson1989}}\newline This
  reference you will find in the \textit{IEEE Xplore} service. Search
  for \textit{johnson xerox star}. 

  Tick the reference and pull down the \textit{Download citations} menu. Select
  \textit{\bt} and \textit{Citation only}. Finally click
  \textit{Download}. Copy the reference to the bib-file.

% \item\textbf{Shneiderman 1983 \cite{shneiderman1983}}\newline Search
%   for \textit{Shneiderman 1983} in \textit{IEEE Xplore} and follow the
%   procedure above.

\item\textbf{Hartson, 1989 \cite{hartson1989}}\newline This reference
  we will find in \textit{ACM Digital Library}.

  A general description of the procedure: Click the title of the
  article in a hit list. The page which appear includes a grey text box to
  the right, marked \textit{Tools and Resources}. At the bottom of
  this box there is a \bt-link. Using this link, you will get the
  \bt-reference in a separate window.

  Move to the \textit{ACM Digital Library} from the library home
  page. Click \textit{Journals and Transactions} under the
  \textit{Browse the digital library} header. You will get a list of
  all the journals of ACM (some called \textit{transactions}). 

  Locate the journal \textit{Computing
    Surveys}. Navigate
  to the journal page, then to the publication archive, to the correct
  issue and finally to the table of contents.

  Study the \bt-reference. Note that ACM abbreviates the journal
  name. Copy the reference to your bib-file and correct the journal
  name.

% \item\textbf{Jacob, 1986 \cite{jacob1986}}\newline Locate the journal
%   \textit{ACM Transactions on Graphics}\footnote{Use Ctrl-f to search
%     within the list.}. See above.

% \item\textbf{Myers, 1995 \cite{myers1995}}\newline Locate the journal \textit{ACM Transactions on
%     Computer-Human Interaction}. See above.

% \item\textbf{Foley, 1987 \cite{foley1987}}\newline Use \textit{Collection
%     of Computer Science Bibliographies} (CCSB). Click \textbf{Search}
%   in the horisontal menu on the start page. Enter \textit{Foley} in
%   the search field and select \textit{author} as qualifier and
%   \textit{1987} as publication year.

% \item\textbf{Foley, 1990 \cite{foley1990}}\newline See
%   above.

% \item\textbf{Shneiderman, 1992 \cite{shneiderman1992}}\newline See
%   above.

% \item\textbf{Stephenson, 1999 \cite{stephenson1999}}\newline See
%   above.

\end{itemize}
The last references are handled by the same routines. As an exercise
in the course aftermath, you may locate and copy the references from
the sources\footnote{You will find \cite{shneiderman1983} in IEEE Xplore,
  \cite{jacob1986,myers1995} in ACM Digital library and
  \cite{foley1990,foley1987,shneiderman1992,stephenson1999} in
  Collection of Computer Science Bibliographies.}. We do not spend
time on this. You may download a readymade bib-file from this location

{\footnotesize\begin{verbatim}
        http://folk.uio.no/knuthe/biblatex/eng/
\end{verbatim}}
\noindent{}Right-click on \textbf{references.bib} and save the file in the \textit{biblatexcourse} directory.

Copy the reference of your thesis into this new
bib-file.
\section{Exercise 2: Using the bib-file with a document}
A rough outline of a \LaTeX\ document using \blt\ is given in
figure~\ref{document}. 

In this exercise you will use text located in the URL

{\footnotesize\begin{verbatim}
        http://folk.uio.no/knuthe/biblatex/eng/
\end{verbatim}}
\noindent{}Right-click on \textbf{userinterface.tex} and save the file in the \textit{biblatexcourse} directory.

Open the file in Emacs. Your job is to replace every doble-*
reference with a proper citation command.

\begin{figure}
\begin{center}
\includegraphics[width=\textwidth]{./documentstructure.jpg}
\caption{Rough outline of a \LaTeX-document using \blt.}\label{document}
\end{center}
\end{figure}

\subsection{Character encoding}
In figure~\ref{document} you see the line
\verb=\usepackage[utf8]{inputenc}=. This means that the character
encoding of both tex and bib files. If you eventually use another
character encoding in the bib file, you must state this by giving the
option \verb=bibencoding= for the biblatex package:

{\footnotesize\begin{verbatim}
    \usepackage[utf8]{inputenc}
    \usepackage[bibencoding=latin1]{biblatex}
\end{verbatim}}
\noindent{}UTF8 is default at IFI.

\subsection{A single or several bib-files}
After loading the \blt\ package you must tell the system which
bib-files you are using.  You do this by giving this command in the
preamble:

{\footnotesize\begin{verbatim}
    \addbibresource{bib-file name}
\end{verbatim}}
\noindent{}The bib extension is required. The command accepts only one file at a
time and you will have to repeat it if you are using several
bib-files. In that case you may want also to collect all your macros
in one single file. This file must then be loaded first:

{\footnotesize\begin{verbatim}
    \addbibresource{macros.bib}
    \addbibresource{library1.bib}
    \addbibresource{library2.bib}
\end{verbatim}}

\subsection{Citing}
You must use a \texttt{cite} command including the reference
identifier as a parameter value when you want to put a
citation into your text: 

{\footnotesize\begin{verbatim}
   \cite{identifier}
\end{verbatim}}
\noindent{}The identifier must be written exactly as it appears in
the bib-file (case sensitive).

If you want to cite several references in the same cite command, it
would look like this --- no space after the comma:

{\footnotesize\begin{verbatim}
   \cite{identifier1,identifier2}
\end{verbatim}}
\noindent{}If you want a prefix or suffix in the citation the cite command accepts two other parameters:

{\footnotesize\begin{verbatim}
   \cite[prefix][suffix]{identifier}
\end{verbatim}}
\noindent{}If you want to refer to a certain page in the cited document, you
would then write something like this:

{\footnotesize\begin{verbatim}
   \cite[See also][47]{identifier
\end{verbatim}}
\noindent{}and this would be printed as \verb=[See also 19, p. 47]=. For more \verb=\cite=-commands, see \cite[][14]{langmyhr13}.

If you want a reference to appear in your reference list
without explicitly citing it, you may write:

{\footnotesize\begin{verbatim}
   \nocite{identifier}
\end{verbatim}}
\noindent{}If you replace \textit{identifier} with *, all of the references in
your bib-file will a appear in your reference list.

\subsection{RefTeX -- Locating and using a reference }
\subsubsection*{Emacs RefTeX mode}
An Emacs add-on (RefTeX) makes it easy to handle
citations\cite{reftex}. The package is recommended also by other
reasons. It handles internal crossreferences in the \LaTeX\ file and makes
it easy to use table of contents in Emacs.

Before setting Emacs in RefTeX mode, you must make a twist. RefTeX
depends on knowing which bib-files you are using. RefTeX supports the
\bt\ logic and will collect the bib-files from the
\kdo{$\backslash$bibliography}-command. 
To tell RefTeX to pick up the bib-file from the
\kdo{$\backslash$addbibresource} you must put the following in your
.emacs file:

{\footnotesize\begin{verbatim}
(setq reftex-bibliography-commands 
   '("addbibresource" "bibliography" "nobibliography")) 
\end{verbatim}}

The snag here is that RefTeX only picks the first bib-file given by
\textbf{addbibresource}. If you use several bib-files, you must give
them in a single \textbf{bibliography} command\footnote{The command
  must be given in the preamble. No bib extension is needed.}:

{\footnotesize\begin{verbatim}
      \bibliography{macros,references}
\end{verbatim}}

\noindent{}You put Emacs in RefTEX mode by giving the command {\footnotesize \verb=M-x reftex-mode=}.\footnote{M-x means first pressing the ESC-button and then the x}

\noindent{}Please note the new top menu (Ref). 

You may set Emacs in RefTeX mode initially by adding these commands to
the .emacs file:

{\footnotesize\begin{verbatim}
   (autoload 'reftex-mode "reftex" "RefTeX Minor Mode" t)
   (autoload 'turn-on-reftex "reftex" "RefTeX Minor Mode" nil)
   (add-hook 'LaTeX-mode-hook 'turn-on-reftex)
\end{verbatim}}


\subsubsection*{Searching the bib-files}
The RefTeX mode will make it possible to locate references by
searching the bib-file with regular expressions.

Place the cursor where you want the cite-command to appear. Select \verb=\cite=
from the Ref menu (or use the command  \verb=C-c [=). Enter the regular
expression plus Return. Eventually you will get a hit list. You may
navigate in the hit list by using the up/down arrows and selecting a
reference by hitting Return. A cite command will appear with the correct
identifier. 

  If you require two or more citations in your cite-command, just
  place the cursor before the last curly brace and repeat the RefTeX
  cite-command.

Now, replace the doble-* citations with proper cite-commands in the
rest of the text.

\subsection{Bibliographic styles}
Styles for citations and references are loaded as an option to the
\blt\ package:

{\footnotesize\begin{verbatim}
      \usepackage[...,
       style=styleoption,
       ...]{biblatex}
\end{verbatim}}
\noindent{}The \kdo{style} gives the styles for both citations and references at
the same time. But they might be differentiated:

{\footnotesize\begin{verbatim}
      citestyle=styleoption-1,
      bibstyle=styleoption-2,
\end{verbatim}}

\noindent{}Here is a list of the most common styles\footnote{To see more styles
  check out \blt\ package manual \cite[][65--70]{biblatex}.}:

\begin{description}
\item[numeric] -- the citation is given as a number in brackets
  ($[23]$). This number points to a numbered list of references. The
  sorting of the references might be decided as an option in the
  preamble. In \textit{this} document \textbf{none} is chosen as the
  sort option, which means that references appears in the order they
  are cited.

\item[alphabetic] -- the citation is a combination of parts of the
  author name and the publication year ($[$KNU99$]$). The references
  are sorted according to this combination.

\item[authortitle] -- the citation is the last name of the author
  followed by the title of the work in italics. To get parentheses
  around the citation you must use the \verb=\parencite= instead of
  \verb=\cite=.\footnote{Or you may type the parenthesis yourself around the \kdo{cite} command.} The references are sorted by name and title.

\item[authoryear] -- the citation is a the last name of the author
  followed by the year of publication. You must use the
  \verb=\parencite= to get parentheses around the citation. The
  references are sorted by name and year.
\end{description}

If you want to use one of the styles APA  (American Psychological Association) or Chicago, check out the local guide.\cite{langmyhr13}

\subsection{Generating the reference list}
Put this command in the \LaTeX-file where you want the reference
list to appear:

{\footnotesize\begin{verbatim}
   \printbibliography
\end{verbatim}}

You then run the following commands in your terminal window:

{\footnotesize\begin{verbatim}
   > pdflatex userinterface.tex
   > biber userinterface
   > pdflatex userinterface.tex
   > pdflatex userinterface.tex
\end{verbatim}}
\noindent{}This is what happens: The first pdflatex command produces some
intermediate files that the \textbf{biber} program uses to pick the
cited references from the bib-files and formating them according to
the styles. The result will be included in the next pdflatex
processing.

If the processing is without error messages, you might inspect the
result in the userinterface.pdf file.

In Linux command mode at IFI you may give the command 
{\footnotesize\begin{verbatim}
   > ltx  userinterface.tex 
\end{verbatim}}
\noindent{}which takes care of all the steps above.

\subsection{Using other styles}
% When changing styles it is a good idea to delete  intermediate
% files  ahead of a new processing:
% 
% {\footnotesize\begin{verbatim}
%    > rm  *.aux *.bbl *.bcf *.blg 
% \end{verbatim}}

Try to change the style to \textit{alphabetic} and process the tex
file. Check the result, both citations and references.

Try changing the style to \textit{authoryear} and process the
file. Please note that you no longer get brackets around your
citation. Change one or more \kdo{$\backslash$cite} commands to
\kdo{$\backslash$parencite} and re-process. Check also the sorting order of the
reference list.

Finally, change back to the \textit{numeric} style. You need not change the \kdo{$\backslash$parencite} command.

\subsection{The reference list header}
The header of the reference list is taken from the value of a
\LaTeX-variable which in turn is  loaded to the \blt-variable
\kdo{bibliography}. 

The name of the variable is \verb=\refname= when using the document
class \textit{article}, and \verb=\bibname=  for the \textit{book}
document class. The default values of these variables are
\textit{References} and \textit{Bibliography} respectively.

You may give these variables new values by using the
\texttt{renewcommand} like this:

{\footnotesize\begin{verbatim}
   \renewcommand{\refname}{Literature}
\end{verbatim}}
\noindent{}You may also state  the header explicitly in the
 \kdo{printbibliography} command: 

{\footnotesize\begin{verbatim}
   \printbibliography[title={Literature}]
\end{verbatim}}
\noindent{}A third possibility is mentioned on page~\pageref{heading}.


\subsection{Including the reference list in the contents}
The reference list will not automatically be included in the table of
contents. To add an entry you give the command \kdo{addcontentsline}
directly after the \kdo{printbibliography} command in the \LaTeX
file. To ensure consistency, define the \verb=\refname= (or
\verb=\bibname= ) variable first:

{\footnotesize\begin{verbatim}
   \renewcommand{\refname}{Literature}
   \printbibliography
   \addcontentsline{toc}{section}{\refname}
\end{verbatim}}
\noindent{}The \textit{toc} tells the system to put a line in the
table-of-contents. The \textit{section} states which document level to
be used and lastly what to put there: the value of the variable
\verb=\refname=.

\subsection{Referring backwards}
When your document grows, it will be difficult to remember just where
you cited a certain reference. 

You may use the package \textit{hyperref}\cite{hyperref} which generates backward
citations. That is, for every reference in the reference list page
numbers are added to show where the reference is cited. To achieve
this, you add the following command in the \LaTeX\ preamble:

{\footnotesize\begin{verbatim}
   \usepackage{hyperref}
\end{verbatim}}
\noindent{}after loading the \blt-package. You must also give the
option \kdo{backref} to \blt. The hyperref package also generates
internal and external links in the document.

Take a look at the reference list in this document to see how it
works. 

\subsection{Section-, type- or chapterwise reference lists}

\subsubsection*{Reference list in each section}
You may generate reference lists for each chapter or section by
marking the start and end of a \textit{refsection} and print the
bibliography within such a section:

{\footnotesize\begin{verbatim}
   \chapter{...}
      \begin{refsection}
      ...
      \printbibliography
      \end{refsection}

   \chapter{...}
      \begin{refsection}
      ...
      \printbibliography
      \end{refsection}
\end{verbatim}}
\noindent{}All citations given within the reference section will be included in
the reference list of that chapter or section.

Any citations appearing outside a reference section will be included
in an overall bibliography.

\subsubsection*{Sectionwise reference lists at the end}
If you want a chapter- or sectionwise bibliography in the end of the
document, you still define reference sections as above, but you do not
usually print them in each section. This is what you might do in the
end of your document:

{\footnotesize\begin{verbatim}
   \printbibheading
   \printbibliography[section=1, heading=subbibliography]
   \printbibliography[section=2, heading=subbibliography]
   \printbibliography[section=3, heading=subbibliography]
   ...
\end{verbatim}}
\noindent{}The first command prints the general bibliography header,
e.g. \textbf{References}.

The commands following will print the references for the relevant
section with the header defined by the variable
\textit{subbibliography}. This header might be defined in the preamble
as follows:\label{heading}


{\footnotesize\begin{verbatim}
   \defbibheading{subbibliography}{%
     \section*{References for Chapter %
     \ref{refsection:\therefsection}}}
\end{verbatim}}
\noindent{}Try this with your userinterface file!
\subsubsection*{Document type or topics reference lists}
You may produce separate lists depending on document type as used in
the bib-file: books, articles etc. Enter the command \verb=\nocite{*}=
and try this in the \texttt{userinterface.tex} file (the commands are already added as comments at the end of the file):

{\footnotesize\begin{verbatim}
   \printbibheading
   \printbibliography[type=book, title={Books}]
   \printbibliography[type=article, title={Articles}]
   \printbibliography[type=manual, title={Manuals}]
   \printbibliography[nottype=book, nottype=article, %
    nottype=manual, title={Other documents}]
\end{verbatim}}
\noindent{}To ensure a running sequence of reference numbers you must use the option \texttt{defernumbers=true} to the biblatex-package.

You may also make a bibliography with the references separated according to different topics entered in the \texttt{keywords} field:

{\footnotesize\begin{verbatim}
   \printbibheading
   \printbibliography[keyword=comm, title={Communication}]
   \printbibliography[keyword=soft, title={Software}]
\end{verbatim}}

\subsection{Changing the value of  bibliography strings}\label{tekster}
There is a lot of text strings you may redefine. Most of them will
contain reasonable default values, so be careful. Here is an example
showing how you can redefine some strings\footnote{You will find a
  list of such strings in section~4.9 in the
  \blt-manual\cite[][see section 4.9, p.215]{biblatex}}:

{\footnotesize\begin{verbatim}
    \DefineBibliographyStrings{english}{%
        urlseen={Seen:},
        bibliography = {Literature},
        references = {Literature},
        page={page},
        pages={pages},
        and={\&},
}
\end{verbatim}}
\noindent{}As you see from the command, such text strings must be associated
with the document language, in this case \textit{english}.


% ************************ end of main text
\vfill

{\footnotesize
\begin{center}
February 2017\\
/undervisning/biblatex-kurs/biblatexbooklet.tex
\end{center}
}
\newpage
(notes)
\newpage
\renewcommand{\refname}{Literature}

\printbibliography\addcontentsline{toc}{section}{\refname}
{\footnotesize
\subsubsection*{A few general Emacs-commands}
\begin{tabular}{|l|l|}\hline
Ctrl-X Ctrl-f &Open\\ \hline
Ctrl-x Ctrl-s &Save\\ \hline
Ctrl-y & Paste\\ \hline
Ctrl-\_& Undo (Ctrl-underscore)\\ \hline
Ctrl-s & Search buffer\\ \hline
Ctrl-c $[$ & Search bib-file (reftex-cite)\\ \hline
\end{tabular}
\subsubsection*{Emacs bib-file commands}
\begin{tabular}{|l|p{7cm}|} \hline
TAB& puts the cursor at the end of the current field\\ \hline
Ctrl-J& puts the cursor at the beginning of the next field\\ \hline
Ctrl-C Ctrl-C &concludes the entering of a reference. You will
be warned about empty mandatory fields. Emacs will propose an
identifier if you have not entered one yourself. Empty optional fields
will be removed.\\ \hline
Syntax check&If you select the menu \kdo{BibTeX-Edit \pil Operating on Buffer or
  Region \pil Validate Entries} a syntax check is performed on the bib-file.\\ \hline
Format entries&If you select the menu \kdo{BibTeX-Edit \pil Operating on Buffer or
  Region \pil Format Entries} the references will be formatted.\\ \hline
\end{tabular}
}
\end{document}
